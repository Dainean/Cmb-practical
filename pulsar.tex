%%%%%%%%%%%%%%%%%%%%%%%%%%%%%%%%%%%%%%%%%
% Journal Article
% LaTeX Template
% Version 1.4 (15/5/16)
%
% This template has been downloaded from:
% http://www.LaTeXTemplates.com
%
% Original author:
% Frits Wenneker (http://www.howtotex.com) with extensive modifications by
% Vel (vel@LaTeXTemplates.com)
%
% License:
% CC BY-NC-SA 3.0 (http://creativecommons.org/licenses/by-nc-sa/3.0/)
%
%%%%%%%%%%%%%%%%%%%%%%%%%%%%%%%%%%%%%%%%%

%----------------------------------------------------------------------------------------
%	PACKAGES AND OTHER DOCUMENT CONFIGURATIONS
%----------------------------------------------------------------------------------------

\documentclass[twoside,twocolumn]{article}

\usepackage{blindtext} % Package to generate dummy text throughout this template 

\usepackage[sc]{mathpazo} % Use the Palatino font
\usepackage[T1]{fontenc} % Use 8-bit encoding that has 256 glyphs
\linespread{1.05} % Line spacing - Palatino needs more space between lines
\usepackage{microtype} % Slightly tweak font spacing for aesthetics

\usepackage[english]{babel} % Language hyphenation and typographical rules

\usepackage[hmarginratio=1:1,top=32mm,columnsep=20pt]{geometry} % Document margins
\usepackage[hang, small,labelfont=bf,up,textfont=it,up]{caption} % Custom captions under/above floats in tables or figures
\usepackage{booktabs} % Horizontal rules in tables

\usepackage{lettrine} % The lettrine is the first enlarged letter at the beginning of the text

\usepackage{enumitem} % Customized lists

\setlist[itemize]{noitemsep} % Make itemize lists more compact

\usepackage{abstract} % Allows abstract customization
\renewcommand{\abstractnamefont}{\normalfont\bfseries} % Set the "Abstract" text to bold
\renewcommand{\abstracttextfont}{\normalfont\small\itshape} % Set the abstract itself to small italic text

\usepackage{titlesec} % Allows customization of titles
\renewcommand\thesection{\Roman{section}} % Roman numerals for the sections
\renewcommand\thesubsection{\roman{subsection}} % roman numerals for subsections
\titleformat{\section}[block]{\large\scshape\centering}{\thesection.}{1em}{} % Change the look of the section titles
\titleformat{\subsection}[block]{\large}{\thesubsection.}{1em}{} % Change the look of the section titles

\usepackage{fancyhdr} % Headers and footers
\pagestyle{fancy} % All pages have headers and footers
\fancyhead{} % Blank out the default header
\fancyfoot{} % Blank out the default footer
\fancyhead[C]{Observations of PSR B0329+54 $\bullet$ December 2016 $\bullet$ Introduction to Radio Astronomy} % Custom header text
\fancyfoot[RO,LE]{\thepage} % Custom footer text

\usepackage{titling} % Customizing the title section

\usepackage{hyperref} % For hyperlinks in the PDF

\usepackage{graphicx}

\graphicspath{ {images/} }
%----------------------------------------------------------------------------------------
%	TITLE SECTION
%----------------------------------------------------------------------------------------

\setlength{\droptitle}{-4\baselineskip} % Move the title up

\pretitle{\begin{center}\Huge\bfseries} % Article title formatting
	\posttitle{\end{center}} % Article title closing formatting
\title{Observations of pulsar B0329+54} % Article title
\author{%
	\textsc{Sander Verdult}\\%\thanks{A thank you or further information} \\[1ex] % Your name
	\normalsize Kapteyn Institute Groningen \\ % Your institution
	\normalsize \href{verdult@fmf.nl}{verdult@fmf.nl} % Your email address
	%\and % Uncomment if 2 authors are required, duplicate these 4 lines if more
	%\textsc{Jane Smith}\thanks{Corresponding author} \\[1ex] % Second author's name
	%\normalsize University of Utah \\ % Second author's institution
	%\normalsize \href{mailto:jane@smith.com}{jane@smith.com} % Second author's email address
}
\date{\today} % Leave empty to omit a date
\renewcommand{\maketitlehookd}{%
	\begin{abstract}
		\noindent Observations of the the pulsar B0329+54 with the Dwingelo radiotelescope, and using the information found to deterine the period, dispersion measure of the interstellar medium, and with some values from literature things such as the age, the distance to our solar system, the magnetic field strength and more.
	\end{abstract}
}

%----------------------------------------------------------------------------------------

\begin{document}
	
	% Print the title
	\maketitle
	
	%----------------------------------------------------------------------------------------
	%	ARTICLE CONTENTS
	%----------------------------------------------------------------------------------------
	
	\section{Introduction}
	%\lettrine[nindent=0em,lines=3]{S}ummary of the aim of the experiment, briefly introducing the field of pulsars, key observational characteristics, and the pulsar that is the target of this observation.\\
%	About pulsars in general\\
	\subsection{Discovery}
	\lettrine[nindent=0em,lines=2]{P}ulsars are stellar remnants. While theoretical models of for neutron stars had been worked out as early as 1939 by Oppenheimer and Volkoff, it took almost thirty years to find the first observational evidence for neutron stars. \\
	In 1967 while carrying out a search for quasars Jocelyn Bell, a graduate student at Cambridge supervised by Antony Hewish) discovered mysterious signals in the data they where collecting: Regular and steady pulses which repeated every 1.337 seconds but which only lasted about 20 ms long. At first those involved dismissed the signals as interference of human origin, but further research showed that their regular appearance and disappearance in the records kept pace with sidereal time (so relative to distant stars) and not to calender time.\\
	While ideas of extraterrestrial intelligent life soon had to be discarded as the source of these signals was fixed with respect to distant stars and did not show any signs of planetary motion. The remaining possibility was that these signals were produced by some new, until now unknown, kind of celestial object of a small radius, this due to the extremely short duration of the pulse). After further research, it was shown that there remained only one hypothetical explanation which was compatable with our knowledge of physics: Pulsars could be rotating neutron stars.
	
	\subsection{What we can measure}
	Directly we can measure the period, the length of the pulses, the intensity and with the right equipment the polarity of the light. From these we can inderectly also determine the magnetic field and properties of the stellar medium between the pulsar and us. Finaly there is the posibility of measuring gravity waves, due to the compactness of the object. Finally, understanding the structure of neutron stars can teach us about many branches of physics and astrophysics.
	\subsection{About our target}
	The pulsar we are investigating is called PSR B0329+54, also known as [WB92] 0329+5426. 
	PSR B0329+54 is the strongest northern pulsar with a distance of 1.06 kpc.\cite{Wang} The Equatorial (J2000.0) coordinates are 53.2472792 Longitude, 54.5786806 Latitude, or alternatively 03h32m59.347s Right Ascencion +and 54d34m43.25s declination. Also known as 0332+5434 in the PSR J catalog.
	\cite{NAD}
	
	\subsection{Dispersion}
	We will also be determining the dispersion measure of the interstellar medium between us and the pulsar. For this reason it is helpful to clarify what a dispersion measure (or DM) actualy is.	Electromagnetic	pulses travelling through a medium propagate with the group velocity. This group velocity varies with frequency, which causes a dispersion in the pulse propagation within a plasma. By interacting with the free electrons in the interstellar medium,	pulses	at	lower frequencies are delayed, causing the arival time of pulsar pulses to shift with frequency. The time delay is related to the DM in the following way: 
	\begin{equation}
	\triangle t=DM*4149/{\nu}^{ 2 }
	\end{equation}
	
	
	%------------------------------------------------
	
	\section{Observations}
	
	\subsection{Telescope}
		The Dwingeloo Radiotelescope started service in 1956, making it one of the oldest telescopes in the world. Up to 1998 it was used for scientific research by ASTRON. By lack of use it started to decay, but by the help of camras it was restored from 2012 to 2014 as a national monument, and is now once more ready for use by radio astronomers, astronomy students and the likes./cite{camras}
	\subsection{Observations}
		Observations of PS B0329+54 where done on Friday 18 November. We used the 25 meter CAMRAS telescope at Dwingeloo at a frequency of 407 to 441 MHZ, focussed around 420 MHz. With constant tracking of the source, we observed the pulsar for a period of 300 seconds, resulting in a datafile that we later accessed over the internet. At our first observation try, some equipment was touched as various bits and there was a fair amount of data loss, causing us to restart the 300 seconds of measurement.
	%Text requiring further explanation\footnote{Bijvoorbeeld}.
	
	
	\section{Data Reduction}
	\subsection{Determining the period}
	Data reduction was done in the web interface of the Astron education website (www.astron.nl/onderwijs), where the data of the observation could be accessed. 
	First we used the dynamical spectrum displayed below to get a rough estimate of the pulse period for our object. This was done by pinning the pulse at a specific wavelength to specific times, and determining the difference. We used the dynamical spectrum between 16 and 18 seconds for our first estimate.

	\begin{figure}[h]
	\centering
	\includegraphics[width=1\linewidth]{dynamical}
	\caption{Dynamical spectrum between 16 and 18 seconds, showing three distinct pulses.The intensity indicates signal strength, the vertical lines show different frequencies. Note that the pulsar signal arrives at different times for different wavelengths}
	\label{Figure 1}
	\end{figure}
	
	
	\begin{table}[h]
		\caption{Pulse measurements}
		\centering
		\begin{tabular}{llr}
			\toprule
			%\multicolumn{2}{c}{Name} \\
			\cmidrule(r){1-2}
			Time (s) & Frequency (MHz) \\
			\midrule
			16.348762 & 417.200781 \\
			17.065662 & 417.200781 \\
			17.780409 & 417.200781\\
			\bottomrule
		\end{tabular}
	\end{table}
	
	Next we used this estimate of the period to do a fast Fourier transform at 420.07 MHz, resulting in the spectrum shown below in figure 2. The spikes are an intiger multiple of the pulse frequency, so by counting the pulses over a larger range, one can make a better a estimate of the pulse time. Dividing 28 pulses over a frequency 39.19 Hz gives us an improved estimate of 0.714444 seconds. 
	
	\begin{figure}[h]
		\centering
		\includegraphics[width=1\linewidth]{spectrum40s}
		\caption{Plot of the fast Fourier transform of the data at 420.07 MHz}
		\label{Figure 2}
	\end{figure}

	Using this improved estimate still showed an offset between the early and latter pulses, showing us that this estimate was not entirely correct either. We started out with figure 3, where the peaks clearly do not overlap, and it took a series of minor changes to get the peaks of the pulse to overlap exactly such as in figure 4, which gave us a final period estimate of 0.714515 seconds
		\begin{figure}[h]
			\centering
			\includegraphics[width=1\linewidth]{Periode_71444}
			\caption{Initial pulse compared to last pulse of the measurement at a period of 0.71444 seconds. Note that the two peaks don't yet overlap.}
			\label{Figure 3}
		\end{figure}
		
		\begin{figure}[h]
			\centering
			\includegraphics[width=1\linewidth]{Periode0_714515}
			\caption{Initial pulse compared to last pulse of the measurement at a period of 0.14515 seconds. Note that the two peak now overlap perfectly.}
			\label{Figure 4}
		\end{figure}
	\newpage
	\subsection{Determining the dispersion measure}
	Having found our period, we can now use our data to determine the time difference in arrival of the pulses different frequencies. Plotting a single pulse at different frequencies shows that the lower frequencies have a higher delay then the higher frequencies due to the electron density between our pulsar and the earth.  \cite{Spingola}
		\begin{figure}[h]
			\centering
			\includegraphics[width=1\linewidth]{Dispersion_adjusted}
			\caption{Dispersion in the signal before adjusting for the Dispersion Measure}
			\label{Figure 5}
		\end{figure}
	
	Measuring these different times at different frequencies, we can use kapteyn-fit to do a least square fitting method on our data, and determine the actual dispersion. With a bit of programming this resulted in a set of data points where the strength of the slope gives us the measured dispersion measure. Adjusting the equation so that we could look for a linear equation, we found a dispersion measure of 26.97  

			\begin{figure}[h]
				\centering
				\includegraphics[width=1\linewidth]{figure_1-3}
				\caption{plotted results of our fitting program}
				\label{Figure 6}
			\end{figure}
			
	Compensating for this dispersion measure should have given us nicely overlapping peaks. Here however we ran in to a bug on the astron site which could not be avoided, and the overlapping peaks where not the ones belonging to our data set. They are still shown in figure 7 however, to make clear what the intent is. 
	\begin{figure}[h]
		\centering
		\includegraphics[width=1\linewidth]{Dedispersed}
		\caption{Compensated for dispersion}
		\label{Figure 7}
	\end{figure}
	
	\subsection{Pulse Profile}
	Now that determined the dispersion measure and the acurate period, we can calculate the pulse profile of PSR B0329+54. By correcting for the dispersion measure and then adding up the profile of all the frequencies in which the pulse is active, we get a single profile covers the pulse in detail. We do believe that though the previous step had a bug, the final result still got determined properly by the web application.
		\begin{figure}[h]
			\centering
			\includegraphics[width=1\linewidth]{Profiel_end}
			\caption{Plot of the pulse profile}
			\label{Figure 8}
		\end{figure}
		
	
	
	%------------------------------------------------

	\section{Results}
	\subsection{Period}
	We determined the period to be P = 0.714515 seconds.
	\subsection{dispersion measure}
	The dispersion measure we found was 26.97.
	
	\subsection{Period change}
	We could not determine a change in the time of our measurements. This in term means that there are a lot of other factors we can't determine, but we will discuss these at the discussions.
	
	%------------------------------------------------
	
	\section{Discussion}
	Summary of what the results tell you about this pulsar, how does it compare with other
	pulsars, what does the dispersion measure tell you about the interstellar medium
	\subsection{Period}
	The period we found ( P = 0.714515 seconds)is a really common period for a pulsar to have. \cite{Wielebinski} Literature gives us a frequency of 1.400 per s, which replies to a period of 0.71429, which is similar to what we measured. \cite{Hobbs} 
	
	\subsection{Period Change}
	Though we could not measure the change in period ourself, we have looked this up in the literature, and we found a value of $\dot{P} = 2.048 \cdot {10}^{-15} s  {s}^{-1}$	\cite{Hobbs}
	
	\subsection{Age of the pulsar}
	If we have the period and the pace at which the period changes, we can also determine the age of the pulsar by the following formula: \cite{Ryan}
		\begin{equation}
	\tau =\frac { 1 }{ 2 } \frac { P }{ \dot { P }  } =\frac { 1 }{ 2 } \frac { 0.71415 }{ 2.048\cdot { 10 }^{ -14 } } 
		\end{equation}
		\begin{equation}
		=1.743{ \cdot 10 }^{ 13 }s\quad =\quad 5.528\cdot { 10 }^{ 5 } year
		\end{equation}
	
	\subsection{Magnetic field of the pulsar}
	With the same information we can also determine the minimum strength of the magnetic field around the pulsar.
	\begin{equation}
	B/tesla\ge 3.3\cdot { 10 }^{ 15 }\sqrt { P\dot { P }  } 
	\end{equation}
	This results in a magnetic field of atleast $3.99\cdot{10}^{8}$ tesla
	
	\subsection{Dispersion measure}
	The dispersion measure we found was 26.97.The one we found in literature was 26.84 \cite{Wang} which is again really close to what we measured. 
	
	\subsection{Distance}
	With the Dispersion measure and a model for the galactic free electron density we can enter the right ascencion and declination of our pulsar, and tell the model to integrate to a dispersion measure of $26.97 pc/{cm}^{3}$ at a frequency of 420 Ghz, in which case we find a distance of 1.115 kpc. Literature\cite{Wang} gives us a paralax distance of $1.06\pm.12$ kpc so this estimate seems correct aswell.
		

	\section{Conclusions}
	\subsection{Errors}
	Important to note is that we have been lax with error margins. Being more precise with these and keeping track of them would have helped us determine how valuable our results are. To properly keep track of the errors in our measurements and calculations, we should have done so from the start. 

	\subsection{Comparison with other pulsars}
	Compared to other pulsars both the magnetic field and the period seem to be part of the norm \cite{Burke}. Beyond the strong intensity, which is most likely due to it's relative close distance, PSR B0329+54 is an ordinary pulsar.

	\subsection{Future experiments}
	More observation times are required to calculate the time derivative of the period. Taking into account effects from the motions of our solar system and earth, we should be able to get an accurate estimation of how quickly the period changes over time. This in turn means that information about the size, age, magnetic field and more can be calculated rather precisely. 
	%----------------------------------------------------------------------------------------
	%	REFERENCE LIST
	%----------------------------------------------------------------------------------------
	
%\bibliographystyle{plain}
%\bibliography{pulsar}

\begin{thebibliography}{99} % Bibliography - this is intentionally simple in this template
	
	
	\bibitem{Camras}Camras\\
	Informatie over de radiotelescoop\\
	http://www.camras.nl/over-de-radiotelescoop/
	
	\bibitem{Wielebinski}
	Richard Wielebinski\\
	The Characteristics of (Normal) Pulsars\\
	Max-planck institut fur Radioastronomie\\
	https://arxiv.org/pdf/astro-ph/0208557v1.pdf
	
	\bibitem{ASTRON}
	astron education website\\
	Data analysis applet\\
	http://astron.nl/onderwijs/
	
	\bibitem{Ryan}
	Sean G. Ryand and Andrew J. Norton\\
	Stellar Evolution and Nucleosynthesis\\
	Cambridge University press, 2010

	\bibitem{Wang}
	Wang, N. and Manchester, R.~N. and Johnston, S. and Rickett, B. and Zhang, J. and Yusup, A. and Chen, M.
	Long-term scintillation observations of five pulsars at 1540 MHz
	journal  = mnras, 2005, volume 358, pages 270-282
	http://mnras.oxfordjournals.org/content/358/1/270.full.pdf

	\bibitem{Hobbs}
	G. Hobbs, A. G. Lyne and M. Kramer
	An analysis of the timing irregularities for 366 pulsars
	http://mnras.oxfordjournals.org/content/402/2/1027
	
	http://adsabs.harvard.edu/abs/2005MNRAS.358..270W

\bibitem{Cordes}
Cordes-Lazio NE2001 Galactic Free Electron Density Model
https://www.nrl.navy.mil/rsd/rorf/ne2001/	

	\bibitem{NAD}
	NASA/IPAC EXTRAGALACTIC DATABASE
   https://ned.ipac.caltech.ed

	\bibitem{Burke}
	B.F.Burke and F.Graham-Smith\\
	An introduction to radio astronomy\\
	Cambridge university press, 2010\\
	
\bibitem{Spingola}
		author   = {{Spingola}},\\
		title    = {Practical 1: Introduction to Observations and Data Analysis},
	
\end{thebibliography}

	
	
	%----------------------------------------------------------------------------------------
	
\end{document}
